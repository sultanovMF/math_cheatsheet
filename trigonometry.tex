\documentclass[twoside, a4paperpt, fleqn]{extarticle}
\usepackage{style}

\tabulinesep=1.2mm
\begin{document}
    \section*{Тригонометрия}

    \subsection*{Основные тригонометрические тождевства}

    \begin{itemize}
        \item $\sin ^{2} \alpha+\cos ^{2} \alpha=1 $
        \item $\sin \alpha=\cos \left(\cfrac{\pi}{2}-\alpha\right) $
        \item $\operatorname{tg}^{2} \alpha+1=\cfrac{1}{\cos ^{2} \alpha} $
        \item $\operatorname{ctg}^{2} \alpha+1=\cfrac{1}{\sin ^{2} \alpha} $
        \item $\operatorname{tg} \alpha \operatorname{ctg} \alpha=1 \operatorname{tg} \alpha=\cfrac{\sin \alpha}{\cos \alpha}$
    \end{itemize}
    \subsection*{Суммы углов}
    \marginpar{Доказательство с комлпексными числами есть в конце файла}
    \begin{itemize}
        \item $ \sin (\alpha \pm \beta)=\sin \alpha \cos \beta \pm \cos \alpha \sin \beta $
        \item $ \cos (\alpha \pm \beta)=\cos \alpha \cos \beta \mp \sin \alpha \sin \beta $
        \item $ \operatorname{tg}(\alpha \pm \beta)=\cfrac{\operatorname{tg} \alpha \pm \operatorname{tg} \beta}{1 \mp \operatorname{tg} \alpha \operatorname{tg} \beta}$
    \end{itemize}
    
    \subsection*{Двойные и тройные углы}

    \begin{itemize}
        \item $ \sin 2 \alpha=2 \sin \alpha \cos \alpha $
        \item $ \cos 2 \alpha=\cos ^{2} \alpha-\sin ^{2} \alpha= 2 \cos ^{2} \alpha-1=1-2 \sin ^{2} \alpha $
        \item $ \operatorname{tg} 2 \alpha=\cfrac{2 \operatorname{tg} \alpha}{1-\operatorname{tg}^{2} \alpha}=\cfrac{2}{\operatorname{ctg} \alpha-\operatorname{tg} \alpha} $
        \item $ \cos 3 \alpha=4 \cos ^{3} \alpha-3 \cos \alpha $
        \item $ \sin 3 \alpha=3 \sin \alpha-4 \sin ^{3} \alpha $
        \item $ \operatorname{tg} 3 \alpha=\cfrac{3 \operatorname{tg} \alpha-\operatorname{tg}^{3} \alpha}{1-3 \operatorname{tg}^{2} \alpha} $
    \end{itemize}

    \subsection*{Произведение функций}
    \begin{itemize}
        \item $ \sin \alpha \cos \beta=\cfrac{1}{2}(\sin (\alpha-\beta)+\sin (\alpha+\beta))$ 
        \item $ \sin \alpha \sin \beta=\cfrac{1}{2}(\cos (\alpha-\beta)-\cos (\alpha+\beta)) $
        \item $ \cos \alpha \cos \beta=\cfrac{1}{2}(\cos (\alpha-\beta)+\cos (\alpha+\beta))$
    \end{itemize}

    \subsection*{Сумма функций}
    \marginpar{$\alpha = A + B, \quad \beta = A - B$ , затем применить формулы суммы углов}
    \begin{itemize}
        \item $\sin \alpha \pm \sin \beta=2 \sin \cfrac{\alpha \pm \beta}{2} \cos \cfrac{\alpha \mp \beta}{2} $
        \item $\cos \alpha \pm \cos \beta=2 \cos \cfrac{\alpha \pm \beta}{2} \cos \cfrac{\alpha \mp \beta}{2} $
        \item $\operatorname{tg} \alpha \pm \operatorname{tg} \beta=\cfrac{\sin (\alpha \pm \beta)}{\cos \alpha \cos \beta} $
        \item $\operatorname{ctg} \alpha \pm \operatorname{ctg} \beta=\cfrac{\pm \sin (\alpha \pm \beta)}{\sin \alpha \sin \beta} $
        \item $\operatorname{ctg} \alpha \pm \operatorname{tg} \beta=\cfrac{\cos (\alpha \mp \beta)}{\sin \alpha \cos \beta} $
        \item $a \sin \alpha+b \cos \alpha= \sqrt{a^{2}+b^{2}} \sin \left(\alpha+\arcsin \cfrac{a}{\sqrt{a^{2}+b^{2}}}\right) $
        \item $\sin x \pm \cos x=\pm \sqrt{2} \sin \left(x \pm \cfrac{\pi}{4}\right) =\pm \sqrt{2} \cos \left(x \mp \cfrac{\pi}{4}\right)$
    \end{itemize}


    \begin{itemize}
        \item $ \cos ^{2} \alpha=\cfrac{1+\cos 2 \alpha}{2} $
        \item $ \sin ^{2} \alpha=\cfrac{1-\cos 2 \alpha}{2}$ 
        \item $ \sin ^{3} \alpha=\cfrac{1}{4}(3 \sin \alpha-\sin 3 \alpha) $
        \item $ \cos ^{3} \alpha=\cfrac{1}{4}(3 \cos \alpha+\cos 3 \alpha) $
        \item $\operatorname{tg} \cfrac{\alpha}{2}=\pm \sqrt{\cfrac{1-\cos \alpha}{1+\cos \alpha}}=\cfrac{\sin \alpha}{1+\cos \alpha}=\cfrac{1-\cos \alpha}{\sin \alpha}$
    \end{itemize}

    \subsection*{Обратные тригонометрические фукнции}
    \begin{itemize}
        \item $ \sin\arccos x = \sqrt{1-x^2} $ 
        \item $ \cos\arcsin x = \sqrt{1-x^2} $ 
        \item $ \sin\arctg x = \cfrac{x}{\sqrt{1+x^2}} $ 
        \item $ \cos\arctg x = \cfrac{1}{\sqrt{1+x^2}} $
        \item $ \tg\arcsin x = \cfrac{x}{\sqrt{1-x^2}} $
        \item $ \tg\arccos x = \cfrac{\sqrt{1-x^2}}{x} $
    \end{itemize}

    
    \begin{tabu}[t]{|c|}
        \hline
            \textbf{Сумма обратных триг. функций} \\
        \hline
            $ \arcsin x - \arcsin y =  \arcsin x + \arcsin(-y) $ \\
            $ \Sigma_{\sin{}} = \arcsin\left(x\sqrt{\mathstrut 1-y^2}+y\sqrt{\mathstrut 1-x^2}\right) $ \\
        \hline
            $ \boxed{\arcsin x + \arcsin y =} $ \\
            $ \left\{ \begin{aligned}
                \Sigma_{\sin{}}, \quad & xy \leqslant 0,\quad & x^2 + y^2 \leqslant 1 \\
                \pi-\Sigma_{\sin{}}, \quad & x > 0, y > 0,\quad  & x^2 + y^2 > 1 \\
                -\pi-\Sigma_{\sin{}}, \quad & x < 0, y < 0,\quad & x^2 + y^2 > 1 \\
            \end{aligned} \right. $ \\
        \hline
            $ \arccos x - \arccos y =  -\pi + \arccos x + \arccos(-y) $ \\
            $ \Sigma_{\cos{}} = \arccos\left(xy+\sqrt{\mathstrut 1-x^2}\sqrt{\mathstrut 1-y^2}\right) $ \\
        \hline
            $ \boxed{\arccos x + \arccos y =} $
            $ \left\{ \begin{aligned}
                \Sigma_{\cos{}}, \quad & x \geqslant -y \\
                2\pi-\Sigma_{\cos{}}, \quad & x < -y
            \end{aligned} \right. $ \\
        \hline
            $ \arctg x - \arctg y =  \arctg x + \arctg(-y) $ \\
            $ \Sigma_{\tg{}} = \arctg\cfrac{x+y}{1-xy} $ \\
        \hline
            $ \boxed{\arctg x + \arctg y =} $
            $ \left\{ \begin{aligned}
                \Sigma_{\tg{}}, \quad & & xy < 1 \\
                \pi+\Sigma_{\tg{}}, \quad & x > 0, \quad & xy>1 \\
                -\pi+\Sigma_{\tg{}}, \quad & x < 0, \quad & xy>1 \\
            \end{aligned} \right. $ \\
        \hline
    \end{tabu}

    \subsection**{Таблицы углов}

    \begin{tabu}[t]{|c|c|c|c|c|c|c|c|c|c|}
        \hline
            \multicolumn{10}{|c|}{\textbf{Таблица значений для стандартных углов}} \\
        \hline
            $ \alpha $ &
            $ 0 $ &
            $ \cfrac{\pi}{6} $ &
            $ \cfrac{\pi}{4} $ &
            $ \cfrac{\pi}{3} $ &
            $ \cfrac{\pi}{2} $ &
            $ \cfrac{2\pi}{3} $ &
            $ \cfrac{3\pi}{4} $ &
            $ \cfrac{5\pi}{6} $ &
            $ \pi $ \\
        \hline
            $ \alpha^{\circ} $ &
            $ 0^{\circ} $ &
            $ 30^{\circ} $ &
            $ 45^{\circ} $ &
            $ 60^{\circ} $ &
            $ 90^{\circ} $ &
            $ 120^{\circ} $ &
            $ 135^{\circ} $ &
            $ 150^{\circ} $ & 
            $ 180^{\circ} $ \\
        \hline
            $ \sin \alpha $ & 	$ 0 $ & 	$ \cfrac{1}{2} $ & 	$ \cfrac{\sqrt{2}}{2} $ & 	$ \cfrac{\sqrt{3}}{2} $ & 	$ 1 $ & 	$ \cfrac{\sqrt{3}}{2} $ & 	$ \cfrac{\sqrt{2}}{2} $ & 	$ \cfrac{1}{2} $ & $ 0 $ \\
        \hline
            $ \cos \alpha $ & 	$ 1 $ & 	$ \cfrac{\sqrt{3}}{2} $ & 	$ \cfrac{\sqrt{2}}{2} $ & 	$ \cfrac{1}{2} $ & 	$ 0 $ & 	$ -\cfrac{1}{2} $ & 	$ -\cfrac{\sqrt{2}}{2} $ & 	$ -\cfrac{\sqrt{3}}{2} $ & $ -1 $ \\
        \hline
            $ \tg \alpha $ & 	$ 0 $ & 	$ \cfrac{\sqrt{3}}{3} $ & 	$ 1 $ & 	$ \sqrt{3} $ & 	--- & 	$ -\sqrt{3} $ & 	$ -1 $ & 	$ -\cfrac{\sqrt{3}}{3} $ & $ 0 $ \\
        \hline
            $ \ctg \alpha $ & 	--- & 	$ \sqrt{3} $ & 	$ 1 $ & 	$ \cfrac{\sqrt{3}}{3} $ & 	0 & 	$ -\cfrac{\sqrt{3}}{3} $ & 	$ -1 $ & 	$ -\sqrt{3} $ & --- \\
        \hline
    \end{tabu}
    
    \vspace*{1 cm}
    
    \begin{tabu}[t]{|c|c|c|c|c|c|c|}
        \hline
            \multicolumn{7}{|c|}{\textbf{Таблица значений для особых углов}} \\
        \hline
            $ \alpha $ &
                $ \cfrac{\pi}{12}  =  15^{\circ} $ &
                $ \cfrac{\pi}{10} = 18^{\circ} $ &
                $ \cfrac{\pi}{5} = 36^{\circ} $ &
                $ \cfrac{3\pi}{10} = 54^{\circ} $ &
                $ \cfrac{2\pi}{5} = 72^{\circ} $ &
                $ \cfrac{5\pi}{12} = 75^{\circ} $ \\
        \hline
            $ \sin \alpha $ &
            $ \cfrac{\sqrt{3}-1}{2\sqrt{2}} $ & 	
            $ \cfrac{\sqrt{5}-1}{4} $ & 	
            $ \cfrac{\sqrt{5-\sqrt{5}}}{2\sqrt{2}} $ & 	
            $ \cfrac{\sqrt{5}+1}{4} $ & 	
            $ \cfrac{\sqrt{5+\sqrt{5}}}{2\sqrt{2}} $ & 	
            $ \cfrac{\sqrt{3}+1}{2\sqrt{2}} $ \\
        \hline
            $ \cos \alpha $ & 	
            $ \cfrac{\sqrt{3}+1}{2\sqrt{2}} $ & 	
            $ \cfrac{\sqrt{5+\sqrt{5}}}{2\sqrt{2}} $ & 	
            $ \cfrac{\sqrt{5}+1}{4} $ & 	
            $ \cfrac{\sqrt{5-\sqrt{5}}}{2\sqrt{2}} $ & 	
            $ \cfrac{\sqrt{5}-1}{4} $ & 	
            $ \cfrac{\sqrt{3}-1}{2\sqrt{2}} $ \\
        \hline
            $ \tg \alpha $ & 	
            $ 2-\sqrt{3} $ & 	
            $ \sqrt{1-\cfrac{2}{\sqrt{5}}} $ & 	
            $ \sqrt{5-2\sqrt{5}} $ & 	
            $ \sqrt{1+\cfrac{2}{\sqrt{5}}} $ & 	
            $ \sqrt{5+2\sqrt{5}} $ & 	
            $ 2+\sqrt{3} $ \\
        \hline
            $ \ctg \alpha $ & 
            $ 2+\sqrt{3} $ & 	
            $ \sqrt{5+2\sqrt{5}} $ & 	
            $ \sqrt{1+\cfrac{2}{\sqrt{5}}} $ & 	
            $ \sqrt{5-2\sqrt{5}} $ & 	
            $ \sqrt{1-\cfrac{2}{\sqrt{5}}} $ & 	
            $ 2-\sqrt{3} $ \\
        \hline
    \end{tabu}

    \subsection*{Тригонометрия и комплексные числа}
    \begin{itemize}
        \item $e^{i\theta}=\cos\theta+i\sin\theta$
        \item $\cos\theta=\cfrac{e^{i\theta}+e^{-i\theta}}{2}$
        \item $\sin\theta=\cfrac{e^{i\theta}-e^{-i\theta}}{2i}$
    \end{itemize}

    \begin{equation*}
        \begin{aligned}
            \cos \left(\theta_{1}+\theta_{2}\right) &=\operatorname{Re}\left(e^{i\left(\theta_{1}+\theta_{2}\right)}\right) \\
            &=\operatorname{Re}\left(e^{i \theta_{1}} e^{i \theta_{2}}\right) \\
            &=\operatorname{Re}\left(\left(\cos \theta_{1}+i \sin \theta_{1}\right)\left(\cos \theta_{2}+i \sin \theta_{2}\right)\right) \\
            &=\cos \theta_{1} \cos \theta_{2}-\sin \theta_{1} \sin \theta_{2}
        \end{aligned}
    \end{equation*}
        
    \begin{equation*}
        \begin{aligned}
            \sin \left(\theta_{1}+\theta_{2}\right) &=\operatorname{Im}\left(e^{i\left(\theta_{1}+\theta_{2}\right)}\right) \\
            &=\operatorname{Im}\left(e^{i \theta_{1}} e^{i \theta_{2}}\right) \\
            &=\operatorname{Im}\left(\left(\cos \theta_{1}+i \sin \theta_{1}\right)\left(\cos \theta_{2}+i \sin \theta_{2}\right)\right) \\
            &=\cos \theta_{1} \sin \theta_{2}+\sin \theta_{1} \cos \theta_{2}
        \end{aligned}
    \end{equation*}

    \begin{equation*}
        \begin{aligned}
        \cos (n \theta)+i \sin (n \theta) &=e^{i n \theta} \\
        &=\left(e^{i \theta}\right)^{n} \\
        &=(\cos \theta+i \sin \theta)^{n}
        \end{aligned}
        \end{equation*}
\end{document}