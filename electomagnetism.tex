\documentclass[twoside, a4paperpt, fleqn]{extarticle}
\usepackage{style}

\begin{document}
\section*{Электромагнетизм}
Закон Кулона:
\begin{equation*}
    F = \frac{1}{4 \pi \varepsilon_0} \frac{\vert Q_1 Q_2 \vert}{r^2}
\end{equation*}
Напряженность электростатического поля:
\begin{equation*}
    \vec{E} = \frac{\vec{F}}{Q_0}
\end{equation*}
Поток вектора напряженности электростатического поля сковзь замкнутую поверхность $S$:
\begin{equation*}
    \label{ phi_E }
    \Phi_E = \oiint_S \vec{E} d \vec{S}
\end{equation*}
Принцип суперпозиции (вообще говоря, экспериментальный факт):
\begin{equation*}
    \vec{E} = \sum \vec{E}_i
\end{equation*}
Электрический момент диполя:
\begin{equation*}
    \vec{p} = |Q| \vec{l}
\end{equation*}
Теорема Гаусса для электростатического поля в вакууме:
\marginpar{см. поток вектора напряженности}
\begin{equation*}
    \oiint \vec{E} d \vec{S} = \frac{1}{\varepsilon_0} \sum Q_i
\end{equation*}
Объемная, поверхностная и линейная плотности заряда:
\begin{equation*}
    \rho = \frac{d Q}{d V}, \quad \sigma = \frac{d Q}{d S}, \quad \tau = \frac{d Q}{ d l}
\end{equation*}
Напряженность поля, создаваемого равномерно заряженной бесконечной плоскостью:
\marginpar[]{Здесь и далее испольузется теорема Гаусса}
\begin{equation*}
    E = \frac{\sigma}{2 \varepsilon_0} \quad (2 E S = \frac{\sigma S}{\varepsilon_0})
\end{equation*}
\marginpar[]{\includegraphics[scale=0.25]{source/physics/application_gauss_theorem1.png}}
\marginpar[]{Применить предыдущую формулу два раза}
Напряженность поля, создаваемого двумя бесконечными паралельными разноименно заряженными плоскостями:
\begin{equation*}
    E = \frac{\sigma}{\varepsilon_0}
\end{equation*}
Напряженность поля, создаваемого объемно заряженным шаром,
\begin{equation*}
E = \frac{1}{4 \pi \varepsilon_0} \frac{Q}{r^2} \quad (r \geq R)
\end{equation*}
\begin{equation*}
    E = \frac{1}{4 \pi \varepsilon_0} \frac{Q}{R^3} r' \quad (r' \leq R)
\end{equation*}
Напряженность поля, создаваемого равномерно заряженным бесконечным цилиндром:
\begin{equation*}
 E = \frac{1}{ 2 \pi \varepsilon_0} \frac{\tau}{r} \quad (r \geq R) \quad E = 0 \quad (r < R)
\end{equation*}
Циркуляция векора напряженности электростатического поля вдоль замнкутого контура $L$:
\marginpar{Работа по замкнутому контуру не производится, потому что электростатическое поля заряда является потенциальным.}
\begin{equation*}
\oint_L \vec{E}{d \vec{l}}= 0
\end{equation*}
Потенциал эллектростатического поля:
\begin{equation*}
    \phi = \frac{U}{Q_0} = \frac{A_{\infty}}{Q_0}
\end{equation*}
Связь между потениалом электростатического поля и его напряженностью:
\begin{equation*}
    \vec{E} = - \nabla \phi
\end{equation*}
Поляризованность:
\begin{equation*}
    \vec{P} = \frac{\sum \vec{p}_i}{V}
\end{equation*}
Связь между векторами $\vec{P}$ и $\vec{E}$:
\begin{equation*}
    \vec{P} = \ae \varepsilon_0 \vec{E}
\end{equation*}
связь между диэлектрической проницаемостью среды и напряженности электростатического поля:
\begin{equation*}
    \vec{D} = \varepsilon_0 \varepsilon \vec{E}
\end{equation*}
Теорема Гаусса для электростатического поля в диэлектрике:
\begin{equation*}
    \oiint \vec{D} d \vec{S} = \sum Q_i
\end{equation*}
Электрическая емкость уединенного проводника:
\marginpar[]{Поскольку $\phi = \int_0^d E d x$, $Q = \sigma S$}
\begin{equation*}
    C = \frac{Q}{\phi}
\end{equation*}
Электроемкость шара:
\begin{equation*}
    C = 4 \pi \varepsilon \varepsilon_0 R
\end{equation*}
Электрическая емкость плоского конденсатора:
\marginpar[]{$\phi = \int_{0}^d E d x$, $Q = $}
\begin{equation*}
    C = \frac{\varepsilon \varepsilon_0 S}{d}
\end{equation*}
% todo написать формулы конденсаторов для других изощренных видов%
Электрическая емкость для параллельного соединения:
\begin{equation*}
    C = \sum C_i
\end{equation*}
Электрическая емкость для последователньо соединенных конденсаторов:
% todo почему? %
\begin{equation*}
    \frac{1}{C} = \sum \frac{1}{C_i}
\end{equation*}
Энергия заряженного уединенного проводника:
\marginpar[]{вывод из мощности $dw = P = UI = \frac{Q d Q}{C dt}$}
\begin{equation*}
    W = \frac{C \phi^2}{2} = \frac{Q \phi} {2} = \frac{Q^2}{2 C}
\end{equation*}
Энергия заряженного конденсатора:
\begin{equation*}
    W = \frac{C \Delta \phi^2}{2} = \frac{Q \Delta \phi} {2} = \frac{Q^2}{2 C}
\end{equation*}
%todo подробно расписать классическую теорию электропроводимости и законы в дифференциальной форме %
Закон Ома для однородного участка цепи:
\begin{equation*}
    I = \frac{U}{R}
\end{equation*}
Мощность тока:
\begin{equation*}
    P = \frac{d A}{d t} = U I = I^2 R = U^2 R
\end{equation*}
Закон Джоуля-Ленца:
\begin{equation*}
    d Q = I U dt = I^2 R dt = \frac{U^2}{R} dt
\end{equation*}
Правила Кирхгофа:
% todo че это за буква для ЭДС?%
\begin{equation*}
    \sum_k I_k = 0; \quad \sum_i I_i R_i = \sum_k \xi_k
\end{equation*}
Связь между индукцией и напряженности магнитного поля:
\begin{equation*}
    \vec{B} = \mu \mu_0 \vec{H}
\end{equation*}
Закон Био-Савара-Лапласа для элемента проводника с током:
\begin{equation*}
    d \vec{B} = \frac{\mu \mu_0}{4 \pi} \frac{I [d \vec{l}, \vec{r}]}{r^3}
\end{equation*}
Магнитная индукция поля прямого тока:
\begin{equation*}
    B = \frac{\mu \mu_0}{4 \pi} \frac{2 I}{2 R}
\end{equation*}
Магнитная индукция поля в центре кругового проводника с током:

%todo вывод этих формул %
\begin{equation*}
    B = \mu \mu_0 \frac{I} {2 R}
\end{equation*}
Сила Лоренца:
\begin{equation*}
    \vec{F} = Q [\vec{v}, \vec{B}]
\end{equation*}
Закон Ампера:
\marginpar{Можно интуитивно вывести формулу Ампера из формулы Лоренца, заметив, что $v = l /t, Q / t = I$}
\begin{equation*}
    d \vec{F} = I [d \vec{l}, \vec{B}]
\end{equation*}
Магнитное поле движущегося заряда:
\begin{equation*}
    \vec{B} = \frac{\mu \mu_0}{4 \pi} \frac{Q [\vec{v}, \vec{r}]}{r^3}
\end{equation*}
Закон полного тока для магнитного поля в вакууме (теорема о циркуляции вектора $\vec{B}$):
\begin{equation*}
    \oint_L \vec{B} d \vec{l} = \mu_0 \sum I_k
\end{equation*}
Магнитная индукция поля внутри соленоида в вакууме:
\marginpar[]{Строим прямоугольник, вне соленоида пренебрегаем, внутри $l B$}
\begin{equation*}
    B = \mu \frac{N I}{l}
\end{equation*}
Поток вектора магнитной индукции сквозь произвольную поверхность $S$:
\begin{equation*}
    \Phi_B = \iint_S \vec{B}{d \vec{S}}
\end{equation*}
Теорема Гаусса для поля с магнитной индукцией $\vec{B}$:
\begin{equation*}
    \iint \vec{B} d \vec{S} = 0
\end{equation*}
Работа по перемещению проводника с током в магнитной поле:
\begin{equation*}
    d A = I(d \Phi_2 - d \Phi_1)
\end{equation*}
Работа по перемещению замкнутого контура в магнитном поле:
\begin{equation*}
    d A = I(d \Phi')
\end{equation*}
Закон Фарадея:
\begin{equation*}
    \xi_i = - \frac{d \Phi}{d t}
\end{equation*}
ЭДС самоиндукции:
\begin{equation*}
    \xi_s = - L \frac{d I}{d t}
\end{equation*}
Индуктивность бесконечно длинного соленоида, имеющего $N$ витков:
\begin{equation*}
    L = \mu_0 \mu \frac{N^2 S}{l}
\end{equation*}
Ток при размыкании цепи:
\begin{equation*}
    I = I_0 \exp(- \frac{t}{\tau})
\end{equation*}
Ток при замыкании цепи:
\begin{equation*}
    I = I + (1 - \exp(- \frac{t}{\tau}))
\end{equation*}
Энергия магнитного поля, связанного с контуром:
\begin{equation*}
    W = \frac{L I^2}{2}
\end{equation*}
Закон полного тока для магнитного поля в веществе:
\begin{equation*}
    \oint_L \vec{B} d \vec{l} = \mu_0 (I + I')
\end{equation*}
Теорема о циркуляции вектора $\vec{H}$:
\begin{equation*}
    \oint_L \vec{H} d \vec{l} = I
\end{equation*}
Плотность тока смещения:
\begin{equation*}
    \vec{j}_{\text{см}} = \frac{\partial \vec{D}}{\partial t} = \varepsilon_0 \frac{\partial \vec{E}}{\partial t} + \frac{\partial \vec{P}}{\partial t}
\end{equation*}
Полная система уравнений Максвелла в интегральной форме:
\begin{equation*}
    \oint_L \vec{E} d \vec{l} = - \iint_S \frac{\partial \vec{B}}{\partial t} d \vec{S}
\end{equation*}
\begin{equation*}
    \oiint_S \vec{D} d \vec{S} = \iiint_V \rho d V
\end{equation*}
\begin{equation*}
    \oint_L \vec{H} d \vec{l} = \iint_S \left( \vec{j} + \frac{\partial \vec{D}}{\partial t} \right) d \vec{S}
\end{equation*}
\begin{equation*}
    \iint_S \vec{B} d \vec{S} = 0
\end{equation*}
Полная система уравнений Максвелла в дифференциальной форме:
\begin{equation*}
    \operatorname { rot } \vec{E} = -\frac{\partial \vec{B}}{\partial t}
\end{equation*}
\begin{equation*}
    \operatorname { div } \vec{D} = \rho
\end{equation*}
\begin{equation*}
    \operatorname{ rot } \vec{H} = \vec{j} + \frac{\partial \vec{D}}{\partial t}
\end{equation*}
\begin{equation*}
    \operatorname { div } \vec{B} = 0
\end{equation*}
Объемная плотность энергии магнитного поля показывает распредление энергии магнитного  поля (конденсатора) к объему. 

$W = \cfrac{L I^2}{2} $ поскольку для соленоида $I = \cfrac{B l}{\mu \mu_0 N}$, то $W = \cfrac{L I^2}{2} = \cfrac{\Phi I}{2} = \cfrac{B S N I}{2} = \cfrac{B S N B l}{2 \mu \mu_0 N} = \cfrac{B^2 V}{2 \mu_0 \mu } = \cfrac{\mu \mu_0 H^2}{2} V$
Объемная плотность энергии электромагнитной волны складывается из объемных плотностей электрического $w_{\text{эл}}$ и магнитного полей $w_{\text{м}}$:
\marginpar[]{объемные плотности магнитного и электрического полей в начальный момент времени одинаковы}
\begin{equation*}
    w = w_{\text{эл}} + w_{\text{м}} = \frac{\varepsilon \varepsilon_0 E^2}{2} + \frac{\mu \mu_0 H^2}{2}
\end{equation*}
Вектор Умова-Пойтинга (плотность потока энергии), напрвлен в сторону распространения электромагнитной волны:
\begin{equation*}
    \vec{S} = [ \vec{E}, \vec{H}]
\end{equation*}
Электромагнитная волна обладает импульсом (W - энергия которую несет волна):
\begin{equation}
    p = \frac{W}{c}
\end{equation}

Длина волны и период колебаний конутра:
\begin{equation*}
    T = \frac{\lambda}{c}
\end{equation*}
\end{document} 