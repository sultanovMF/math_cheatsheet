\documentclass[twoside, a4paperpt]{extarticle}
\usepackage{style}


\begin{document}
\section*{Таблица БМФ при $x \rightarrow 0$}
\begin{multicols}{2}
    \begin{itemize}
        \item $\sin x \sim x$
    
        \item $\tan x \sim x$
    
        \item $\arcsin x \sim x$
    
        \item $\arctan x \sim x$
    
        \item $\cos x \sim 1-\cfrac{x^{2}}{2}$
    
        \item $\ln (1+x) \sim x$
    
        \item $\log _{a}(1+x) \sim \cfrac{x}{\ln a}$
    
        \item $e^{x}-1 \sim x$
    
        \item $a^{x}-1 \sim x \ln a$
    
        \item  $(1+x)^{a}-1 \sim a x$
    \end{itemize}
\end{multicols}

\section*{Формулы из приложений определённого интеграла}

\subsection*{Площадь}
\begin{itemize}

    \item  Типикал $S=\displaystyle \int_{a}^{b} y(x) d x$

    \item  Типикалі, но с двумя кривыми $S=\displaystyle \int_{a}^{b}\left(y_{2}(x)-y_{1}(x)\right) d x$

    \item  Параметрическое

    \begin{itemize}
        \item $ S=-\displaystyle \int_{T_{0}}^{T} y(t) x^{\prime}(t) d t $

        \item $ S=\displaystyle \int_{T_{0}}^{T} x(t) y^{\prime}(t) d t  $
    
        \item $ S=\cfrac{1}{2} \displaystyle \int_{T_{0}}^{T}\left(x(t) y^{\prime}(t)-y(t) x^{\prime}(t)\right) d t $
    \end{itemize}



    \item  Явная полярка $S=\cfrac{1}{2} \displaystyle \int_{\alpha}^{\beta} r^{2}(\varphi) d \varphi$

    \item  Параметрическая полярка $S=\cfrac{1}{2} \displaystyle \int_{T_{0}}^{T} r^{2}(t) \varphi^{\prime}(t) d t$
\end{itemize}
\subsection*{Вычисление длины дуги}
\begin{itemize}
    \item  Декартовые $L=\displaystyle \int_{a}^{b} \sqrt{1+\left(f^{\prime}(x)\right)^{2}} d x$
    \item  Параметр $L=\displaystyle \int_{t_{0}}^{t_{1}} \sqrt{\left(\varphi^{\prime}(t)\right)^{2}+\left(\psi^{\prime}(t)\right)^{2}} d t$
    \item  Полярка $L=\displaystyle \int_{\alpha}^{\beta} \sqrt{\left(r^{\prime}(\varphi)\right)^{2}+(r(\varphi))^{2}} d \varphi .$
\end{itemize}

\subsection*{Вычисление объемов тел вращения}
\begin{itemize}
    \item $V=\displaystyle \int_{a}^{b} S(x) d x$
    \item $V_{O X}=\pi \displaystyle \int_{a}^{b} f^{2}(x) d x$
    \item $V_{O Y}=2 \pi \displaystyle \int_{a}^{b} x f(x) d x$
    \item Сектор в полярке $V=\cfrac{2}{3} \pi \displaystyle \int_{\alpha}^{\beta} r^{3}(\varphi) \sin \varphi d \varphi$ 
\end{itemize}

\subsection*{Площадь поверхности вращения}
\begin{itemize}
    \item $S_{O X}=2 \pi \displaystyle \int_{a}^{b}|f(x)| d l=2 \pi \displaystyle \int_{a}^{b} f(x) \sqrt{1+(f \prime(x))^{2}} d x$
    \item $S_{O X}=2 \pi \displaystyle \int_{t_{0}}^{t_{1}} \psi(t) \sqrt{(\psi \boldsymbol{\gamma}(t))^{2}+(\varphi \prime(t))^{2}} d t$
    \item $S_{O X}=2 \pi \displaystyle \int_{\alpha}^{\beta} r(\varphi)|\sin \varphi| \sqrt{r^{2}(\varphi)+(r \prime(\varphi))^{2}} d \varphi$
\end{itemize}

\end{document}